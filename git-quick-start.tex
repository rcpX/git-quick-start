%\documentclass[11pt,a4paper]{article}
\documentclass[11pt]{article}

%\usepackage{a4wide}
\usepackage[margin=1.4in]{geometry}

\usepackage{graphicx}
%\usepackage[german, english]{babel}
%\usepackage{tabularx}
%\usepackage{cancel}
%\usepackage{multirow}
%\usepackage{supertabular}
%\usepackage{algorithmic}
%\usepackage{algorithm}
%\usepackage{amsthm}
%\usepackage{float}
%\usepackage{subfig}
%\usepackage{rotating}
\usepackage{amsmath}

\usepackage[T1]{fontenc} % Font encoding
\usepackage{charter} % Serif body font
\usepackage[charter]{mathdesign} % Math font
\usepackage[scale=0.9]{sourcecodepro} % Monospaced fontenc
\usepackage[lf]{FiraSans} % Sans-serif font

\usepackage{listings}
\lstset{
    basicstyle=\ttfamily
}
\usepackage{hyperref}

\usepackage{soul} % for st strikethrough command

%\usepackage[round]{natbib}
\usepackage[natbib=true, style=authoryear, bibstyle=authoryear-comp, 
maxbibnames=10,
maxcitenames=2, backend=bibtex8]{biblatex}
\bibliography{/Users/adamsgaard/articles/own/BIBnew.bib}


\begin{document}

\title{A quick-start guide to Git}

\author{Anders Damsgaard\\\url{https://adamsgaard.dk}, \url{andersd@riseup.net}}
\date{{\small Last revision: \today}}

\maketitle

\section{What is Git?}
Git is the most popular command-line tool for version control.  It is most 
commonly used to track changes to plain-text files such as source code.  When a 
software project is initialized as a Git \emph{repository}, the history of the 
tracked files are recorded through a series of changes.  When the user performs 
changes to the tracked files, she can choose to \emph{commit} these changes.  
Git repositories can be managed through online services such as 
Github\footnote{\url{https://github.com}}, which allows the changes to be 
synchronized between multiple contributers and end users.  The same repository 
can contain multiple versions of the same files.  These coexisting versions are 
called \emph{branches}.

Git usage is typically very verbose and by design weighs explicitness over 
convenience.  This ensures that its default behavior does not lead to unintended 
outcomes.

\section{Installation}
Please refer to the official documentation\footnote{%
\url{https://git-scm.com/book/en/v2/Getting-Started-Installing-Git}} for 
platform-specific instructions on how to install Git.
My prefered installation method on OS X is through 
Homebrew\footnote{\url{http://brew.sh}}:
\begin{lstlisting}
    $ brew install git
\end{lstlisting}
On Debian-based systems, Git can be installed through the advanced package tool:
\begin{lstlisting}
    $ apt-get install git
\end{lstlisting}

Git has excellent built-in documentation through its man page (i.e. \texttt{man 
    git}).  For documentation on a sub-command such as \texttt{git add}, see its 
documentation with \texttt{man git-add}. For more complex tasks, I recommend 
referring to a Git handbook or searching the web (in that order).

\section{Getting started}
Before using Git for version control, it is a good idea to record some 
information about yourself. This makes it is easy to see who specific commits 
can be attributed to when working on projects with multiple contributors.  The 
user information is stored in a plain-text file in the home directory 
(\texttt{\textasciitilde/.gitconfig}), and can be created with the following 
commands:
\begin{lstlisting}
    $ git config --global user.name "John Doe"
    $ git config --global user.email "john-doe-farms@aol.com"
\end{lstlisting}

\section{Initializing a repository}
In order to track changes to files in a directory, the directory needs to be 
initialized as a repository.  Let's say that we want to track the changes to a 
file \texttt{arithmetic.c} which located in the directory 
\texttt{\textasciitilde/src/calculator}.  We start off by initializing the 
directory as a repository:
\begin{lstlisting}
    $ cd ~/src/calculator
    $ git init
    Initialized empty Git repository in ~/src/calculator/.git/
\end{lstlisting}
Git lets us know that the directory is initialized as a new repository, and that 
the hidden sub-directory \texttt{.git} is used for the files related to the 
version control\footnote{This directory contains many interesting files, and I 
    encourage you to explore it when you are more familiar with Git.}.

It is important to realize that just because you initialize a directory as a Git 
repository, the files and subdirectories are \emph{not automatically tracked}!  
You need to manually specify which files inside of the directory you want to 
include in the version control system.  There are several important reasons for 
this.  As an example, during the compilation step many compilers create object 
files which are linked together to create the final executable binaries.  These 
object files are created in machine code, and can have a significant size on the 
file system.  If Git automatically recorded the changes and versions to all 
files inside of the repository, it would save all changes in the binary object 
files, which are of no use since they can be readily reconstructed from the 
source code.

\section{Creating a local copy of an online repository}
If you want to create a local copy of an online repository you can download a 
clone of it in its current stage to your local file system:
\begin{lstlisting}
    $ git clone git://github.com/john-doe/tractor-simulation
\end{lstlisting}
This will checkout the online repository on Github into a corresponding 
directory in the local directory.  You can also choose to clone over other 
protocols such as SSH or HTTPS if you prefer.

The local repository will remember where it was cloned from.  If you have write 
permissions to the online repository, you can upload your local commits using 
\texttt{git push}.

\section{Adding files and commiting changes}
To add a file to the version-control system inside a repository, use the 
following command:
\begin{lstlisting}
    $ git add arithmetic.c
\end{lstlisting}

One or more changes to the tracked files in a repository must be accompanied by 
a \emph{commit message}.  The commit message should ideally be short and 
descriptive of the changes that are contained in the commit.  The commit 
messages are logged.  It should be easy for a user to glance through the log of 
commit messages and understand the changes without reading the changes to the 
files themselves.  This is what sets version-control systems aside from 
automatic backups.  Only the user herself has the ability to identify when a set 
of changes are complete and significant, and can formulate a meaningful 
description of the changes in their respective context.

In case you have forgotten what you have changed in a file, use the following 
command:
\begin{lstlisting}
    $ git diff -- arithmetic.c
\end{lstlisting}

To commit all files which have been added to the repository, you can use the 
following command:
\begin{lstlisting}
    $ git commit -m "First commit of arithmetic.c"
\end{lstlisting}
If you ommit the \texttt{-m} flag and message string (i.e. simply type 
\texttt{git commit}), Git will open your favorite command-line 
editor\footnote{You can set which editor you want to use using the 
    \texttt{EDITOR} environment variable in e.g.  
    \texttt{\textasciitilde/.bashrc}.}.  You then write the commit message in 
the editor, and finalize the commit by saving and exiting the file.

If you perform subsequent edits to the file, you need to commit the new changes 
once again.  We can either once again add the file and commit the changes:
\begin{lstlisting}
    $ git add arithmetic.c
    $ git commit -m "Implemented multiplication"
\end{lstlisting}
Alternatively, since the file \texttt{arithmetic.c} is already added to the 
repository, you can commit \emph{all} changes to \emph{all} tracked files in the 
repository with a single command:
\begin{lstlisting}
    $ git commit -a -m "Implemented multiplication"
\end{lstlisting}
Are you not sure which files changed since the last commit? Git can show you an 
overview of the current state:
\begin{lstlisting}
    $ git status
\end{lstlisting}

\section{Inspecting repository changes and reverting to a previous commit}
Git can show you an overview of all recorded changes in the repository:
\begin{lstlisting}
    $ git log
    commit 2745f1e3b4803f1c8728089667a18f3178cd18dc
    Author: John Doe <john-doe-farms@aol.com>
    Date:   Fri Sep  2 10:22:59 2016 -0700

        Implemented multiplication

    commit 3329dfa1b6bfecc00353d1e9db50bcab9fb41521
    Author: John Doe <john-doe-farms@aol.com>
    Date:   Fri Sep  2 10:22:13 2016 -0700

        First commit of arithmetic.c

\end{lstlisting}
Git can show the changes to the files between any two commits:
\begin{lstlisting}
    $ git diff 3329dfa1b 2745f1e3b
    diff --git a/arithmetic.c b/arithmetic.c
    index e69de29..523d72b 100644
    --- a/arithmetic.c
    +++ b/arithmetic.c
    @@ -0,0 +1,4 @@
    +double multiply(double x, double y)
    +{
    +    return x * y;
    +}
\end{lstlisting}

In case you want to roll back your most recent changes and revert the repository 
to a stage corresponding to an earlier commit.  Each commit has an uniquely 
identifying string which is shown with the above command.  To revert you need to 
supply the first 9 characters of this string:
\begin{lstlisting}
    $ git checkout 3329dfa1b
\end{lstlisting}
This will revert any changes contained in subsequent commits.  In case you 
change your mind and want to go back to the most recent commit, use:
\begin{lstlisting}
    $ git revert HEAD
\end{lstlisting}
The special string \texttt{HEAD} refers to the most recent commit.

\section{Branching and merging}
Git allows you to have multiple versions (branches) of the same repository.  The 
first step is to create a new branch and give it a suitable name:
\begin{lstlisting}
    $ git checkout -b new_interface
\end{lstlisting}
Subsequent commits are staged to the new branch \texttt{new\_interface}.  You 
can see which branches are present in the repository with \texttt{git branch}:
\begin{lstlisting}
    $ git branch
      master
    * new_interface
\end{lstlisting}
where \texttt{master} is the original branch.  The asterisk denotes what current 
branch is active.  You can switch between branches, which will automatically 
apply all relevant patches to the affected files:
\begin{lstlisting}
    $ git checkout master
\end{lstlisting}
To delete a branch, use \texttt{git branch -d new\_interface}.  To merge another 
branch into your current active branch, use:
\begin{lstlisting}
    $ git merge new_interface
\end{lstlisting}
When merging, all commits and file changes performed in a branch are applied to 
the currently active branch.

\section{Ignoring files}
Many compilers create auxillary files which are never relevant to track in a 
version-control system, but clutter your repository overview when using commands 
such as \texttt{git status} or \text{git commit -a}.  You can specify which 
files Git should ignore by their filename in a file at the root of the 
repository in a file named \texttt{.gitignore}.
For a repository containing C code, an example \texttt{.gitignore} file could 
contain:
\begin{lstlisting}
    *.o
\end{lstlisting}
This will ignore all object files.  For a \LaTeX~repository the file could 
contain:
\begin{lstlisting}
    *.aux
    *.glo
    *.idx
    *.log
    *.toc
    *.ist
    *.acn
    *.acr
    *.alg
    *.bbl
    *.blg
    *.dvi
    *.glg
    *.gls
    *.ilg
    *.ind
    *.lof
    *.lot
    *.maf
    *.mtc
    *.mtc1
    *.out
    *.xdy
    *.synctex.gz
\end{lstlisting}
It is up to you to specify the contents of the \texttt{.gitignore} file.  Maybe 
your program generates output files, which should not be tracked. Simply add 
their names or file type to the \texttt{.gitignore} file and never encounter 
them in your Git workflow again.

\section{Extra: Useful shell aliases}
I like to bind short aliases to the most commonly Git commands. I do this by 
appending the following to the \texttt{rc} file of my shell 
(\texttt{\textasciitilde/.zshrc} or \texttt{\textasciitilde/.bashrc}):
\begin{lstlisting}
    alias gs='git status | less'
    alias gl='git log --graph --oneline --decorate --all'
    alias ga='git add'
    alias gd='git diff --'
    alias gc='git commit -v'
    alias gca='git commit --all --verbose'
    alias gp='git push'
    alias gpu='git pull'
    alias gcgp='git commit --verbose && git push'
    alias gcagp='git commit --all --verbose && git push'
\end{lstlisting}
Using these aliases I can quickly add a file (\texttt{ga file.c}).  
Alternatively, I can quickly commit all changes to all files that are already 
tracked in the repository (\texttt{gca}).
With \texttt{gl} I can quickly see the commit tags and commit messages in short 
form, and scroll up and down with \texttt{j} and \texttt{k} or the arrow keys.  
\texttt{gs} gives me a quick overview of the changes in the current repository, 
and uses the same keys as \texttt{gl} for scrolling.

\end{document}

